\documentclass[a4paper,12pt,titlepage,finall]{article}

\usepackage[T1,T2A]{fontenc}     % форматы шрифтов
\usepackage[utf8x]{inputenc}     % кодировка символов, используемая в данном файле
\usepackage[russian]{babel}      % пакет русификации
\usepackage{tikz}                % для создания иллюстраций
\usepackage{pgfplots}            % для вывода графиков функций
\usepackage{geometry}		 % для настройки размера полей
\usepackage{indentfirst}         % для отступа в первом абзаце секции
\usepackage{multirow}            % для таблицы с результатами

% выбираем размер листа А4, все поля ставим по 3см
\geometry{a4paper,left=30mm,top=30mm,bottom=30mm,right=30mm}

\setcounter{secnumdepth}{0}      % отключаем нумерацию секций

\usepgfplotslibrary{fillbetween} % для изображения областей на графиках

\begin{document}
% Титульный лист
\begin{titlepage}
    \begin{center}
	{\small \sc Московский государственный университет \\имени М.~В.~Ломоносова\\
	Факультет вычислительной математики и кибернетики\\}
	\vfill
	{\Large \sc Отчет по заданию №1}\\
~\\
	{\large \bf <<Методы сортировки>>}\\ 
~\\
	{\large \bf Вариант 2 / 1 / 2 / 5}
    \end{center}
    \begin{flushright}
	\vfill {Выполнил:\\
	студент 104 группы\\
	Киперь~Д.~Ю.\\
~\\
	Преподаватели:\\
	Цыбров~Е.~Г.\\
    Кулагин~А.~В.}
    \end{flushright}
    \begin{center}
	\vfill
	{\small Москва\\2025}
    \end{center}
\end{titlepage}

% Автоматически генерируем оглавление на отдельной странице
\tableofcontents
\newpage

\section{Постановка задачи}
{\large Задание требовало от меня сделать следующее:}

\begin{itemize}
    \item написать генератор массивов длин 10, 100, 1000, 10000, элементами которых будут числа типа \textbf{long~long~int}; 
    \item сортировать числа в массивах в порядке неубывания;
    \item написать два алгоритма сортировок: Сортировка простым выбором и Пирамидальная сортировка;
    \item сделать подсчет количества перестановок и сравнений в каждой из сортировок;
    \item провести эксперементальное сравнение двух алгоритмов сортировки;
\end{itemize}
\newpage


\section{Структура программы и определение функций}

Моя программа разделена на два модуля:
\begin{itemize}
    \item генератор массивов~(с случайными элементами типа \textbf{long long int}) для сортировки
    \item сортировщик массивов
\end{itemize}

В первом модуле, как следует из названия, создаются и записываются в файлы массивы для сортировки.
В другом модуле реализовано считывание числе из файлов и запись их в массив, с последующей сортировкой
двумя типами сортировок: пирамидальной и простым выбором. Их результаты записываются в файлы
в каталог \textbf{output/<<name of sort>>}.

\subsection{Генерация массивов}

Сначала объявим заголовочные файлы и константы.
\begin{verbatim}
    #include <stdlib.h>
    #include <string.h>
    #include <stdio.h>
    #include <limits.h>
    #include <time.h>
    
    #define INPUT_CATALOG "files/inputs/"
    #define OUTPUT_CATALOG "files/outputs/"
    
    #define SMALL_NUM 10
    #define LITTLE_NUM 100
    #define MIDDLE_NUM 1000
    #define BIG_NUM 10000
    
    #define MAX_BUF_SIZE 80
    
    #define BORDER 100000
\end{verbatim}

Процесс генерации массивов включает в себя создание массива заданной длины с элементами типа \textbf{long long int}. \\
Так как длина массива может доходить до 10000, то для удобства он будет сохранятся в отдельный файл.

\begin{verbatim}
    FILE* file_open(char* filename, char* mode) {
        FILE* file = fopen(filename, mode);
        if (file == NULL) {
            if (strcmp("w", mode) == 0) {
                printf("Ошибка при открытии файла на запись!!\n");
            }

            if (strcmp("r", mode) == 0) {
                printf("Ошибка при открытии файла на чтение!!\n");
            }

            return NULL;
        }

        return file;
    }
\end{verbatim}

Данная функция написана в генераторе чтобы при ошибке создания файла, было выведено сообщение в консоли.

\begin{verbatim}
    void print_list(long long int* arr, int n) {
        for (int i = 0; i < n; i++) {
            printf("%lld ", arr[i]);
        }
    }
\end{verbatim}

Эта вспомогательная функция выводит в консоль элементы массива (была написана для упрощенной отладки).

\begin{verbatim}
    long long int get_rand_number() {
        return ((long long int)rand() << 32) | rand();
    }
\end{verbatim}

Эта функция генерирует случайное число, которое будет использовано для заполнения массива.

\begin{verbatim}
    long long int* generate_array(int n) {
        long long int* arr = (long long int*) malloc(sizeof(long long int) * n);

        for (int i = 0; i < n; i++) {
            arr[i] = get_rand_number();

            while (arr[i] >= LLONG_MAX - BORDER || arr[i] <= LLONG_MIN + BORDER) {
                arr[i] = get_rand_number();
            }
        }

        return arr;
    }

    long long int* generate_sorted_array(int n) {
        long long int* arr = (long long int*) malloc(sizeof(long long int*) * n);

        arr[0] = get_rand_number();
        for (int i = 1; i < n; i++) {
            int ost = get_rand_number() % 10;
            
            while (ost == 0) {
                ost = get_rand_number() % 10;
            }

            arr[i] = arr[i - 1] + ost;
        }

        return arr;
    }
\end{verbatim}

\newpage

\begin{verbatim}
    long long int* generate_rotated_sorted_array(int n) {
        long long int* arr = (long long int*) malloc(sizeof(long long int*) * n);

        arr[0] = get_rand_number();
        for (int i = 1; i < n; i++) {
            int ost = get_rand_number() % 10;
            
            while (ost == 0) {
                ost = get_rand_number() % 10;
            }

            arr[i] = arr[i - 1] - ost;
        }

        return arr;
    }
\end{verbatim}

Эти три функции обеспечивают генерацию трех массивов (в несортированном, сортированном, сортированном в обратном порядке).
Память под каждый массив выделяется динамически, и в массив помещаются случайные числа типа \textbf{long~long~int}.

\begin{verbatim}
    void create_file(
        char* filename,
        char* filname_sorted,
        char* filename_rotated_sorted,
        int n
    ) {
        long long int* arr = generate_array(n);
        long long int* sorted_arr = generate_sorted_array(n);
        long long int* rotated_sorted_arr = generate_rotated_sorted_array(n);

        FILE* file = file_open(filename, "w");
        FILE* file_sorted = file_open(filname_sorted, "w");
        FILE* file_rotated_sorted = file_open(filename_rotated_sorted, "w");

        for (int i = 0; i < n; i++) {
            fprintf(file, "%lld\n", arr[i]);
        }

        for (int i = 0; i < n; i++) {
            fprintf(file_sorted, "%lld\n", sorted_arr[i]);
        }

        for (int i = 0; i < n; i++) {
            fprintf(file_rotated_sorted, "%lld\n", rotated_sorted_arr[i]);
        }

        fclose(file);
        fclose(file_sorted);
        fclose(file_rotated_sorted);

        free(arr);
        free(sorted_arr);
        free(rotated_sorted_arr);
    }
\end{verbatim}

Эта функция принимает имя файлов (несортированного, отсортированного и сортированного в обратном порядке) и размер каждого из массивов.
После создаются файлы, и происходит запись данных из массивов в файлы. В самом конце чистим память.

\begin{verbatim}
    void generate_data() {
        int num_params[] = { SMALL_NUM, LITTLE_NUM, MIDDLE_NUM, BIG_NUM };
        int len = sizeof(num_params) / sizeof(num_params[0]);

        for (int i = 0; i < len; i++) {
            char filename[MAX_BUF_SIZE];
            char filename_sorted[MAX_BUF_SIZE];
            char filename_rotated_sorted[MAX_BUF_SIZE];

            sprintf(
                filename,
                "%sunsorted%d.txt",
                INPUT_CATALOG,
                num_params[i]
            );
            sprintf(
                filename_sorted,
                "%ssorted%d.txt",
                INPUT_CATALOG,
                num_params[i]
            );
            sprintf(
                filename_rotated_sorted,
                "%srotated_sorted%d.txt",
                INPUT_CATALOG,
                num_params[i]
            );

            create_file(
                filename,
                filename_sorted,
                filename_rotated_sorted,
                num_params[i]
            );
        }
    }
\end{verbatim}

Одна из основных функций в генераторе, ибо явлется корнем создания файлов.
Она отвечает за создание имен файлов, и вызов функции, которая создает файл с таким именем и числом параметров.

\begin{verbatim}
    int main(void) {
        srand(time(NULL));

        generate_data();

        return 0;
    }
\end{verbatim}

Эта функция является входной точкой всего генератора.
В ней мы подключаем возможность генерировать случайные числа,
а затем вызываем функцию, которая генерирует данные и записывает их в файлы.

\newpage

\subsection{Сортировки и вспомогательные функции}

Моей задачей было реализовать сортировку простым выбором и пирамидальную сортировку.
Выведем заголовочные файлы и константы, которые я использовал:
\begin{verbatim}
    #include <stdio.h>
    #include <stdlib.h>
    #include <string.h>
    
    #define INPUT_CATALOG "files/inputs/"
    #define OUTPUT_CATALOG "files/outputs/"
    
    #define SMALL_NUM 10
    #define LITTLE_NUM 100
    #define MIDDLE_NUM 1000
    #define BIG_NUM 10000
    
    #define UNSORTED "unsorted"
    #define SORTED "sorted"
    #define SORTED_ROTATED "rotated_sorted"
    
    #define MAX_BUF_SIZE 1000
    #define MEAN_BUF_SIZE 200
\end{verbatim}

\subsection{Сортировка простым выбором}

Суть сортировки простым выбором заключается в том, что сначала мы проходимся по всему массиву, находим
в нем максимальный элемент и меняем его с элементом, стоящим в конце массива. Далее сдвигаемся с конца массива на 1 влево
и проделываем то же самое, только максимальный эл-т ищется до $n - 2$ индекса (включительно),
а так же меняется местами с эл-ом с индексом $n - 2$. Реализация такого алгоритма приложена ниже.

\begin{verbatim}
    void selection_sort(long long int* arr, int n) {
        int j = n - 1;
        int max_i = 0;

        int compare = 0, transp = 0;

        for (int i = 0; i < n; i++) {
            for (int q = 0; q <= j; q++) {
                compare++;

                if (arr[q] > arr[max_i]) {
                    max_i = q;  
                }
            }

            swap(&arr[max_i], &arr[j]);
            transp++;

            j--;
            max_i = 0;
        }

        printf(
            "Selection sort, размер массива - %d\n\tКоличество сравнений - %d,
            количество перестаовок - %d\n",
            n,
            compare,
            transp
        );
    }
\end{verbatim}


\newpage

\subsection{Пирамидальная сортировка}

Структура данных (бинарная) пирамида представлет
собой объект~-массив, который можно рассматривать как почти полное бинарное дерево.
Каждый узел этого дерева соответствует эл-ту массива. Дерево полностью на всех уровнях 
за исключением возможно наинизшего, который заполняется слева направо.
Корнем дерева является A[1] (где 1~-- порядковый номер в массиве а не индекс). Для заданного 
индекса (нумерация с 1) можно легко вычислить:
\begin{itemize}
    \item его родителя \textbf{A[i / 2]}
    \item его левого ребенка \textbf{A[2 * i]}
    \item его правого ребенка \textbf{A[2 * i + 1]}
\end{itemize}

На большинстве компьютеров (особенно в наше время) операция $2i$ выполняется с помощью одной
команды, побитового сдвига влево на 1 бит, аналогично 
операция $2i + 1$ также выполняется очень быстро, операция получения родителя выполняется
сдвигом $i$ на бит вправо.\\
В варианте моей пирамидальной сортировки выполняется свойтство (свойство неубывающих пирамид) \\
\begin{center}
    $A[Parent(i)] <= A[i]$
\end{center}

Так как в задании было сказано, что отсортировать нужно в неубывающем порядке.
Таким образом наименьший элемент пирамиды находится в ее корне.

\begin{verbatim}
    void heapify(long long int* arr, int n, int i, int* compare, int* transp) {
        int largest = i;
        int left = 2 * i + 1;
        int right = 2 * i + 2;

        if (left < n) {
            (*compare)++;

            if (arr[left] > arr[largest]) {
                largest = left;
            }
        }

        if (right < n) {
            (*compare)++;

            if (arr[right] > arr[largest]) {
                largest = right;
            }
        }

        if (largest != i) {
            swap(&arr[i], &arr[largest]);
            (*transp)++;

            heapify(arr, n, largest, compare, transp);
        }
    }    
\end{verbatim}

На каждом шаге определяется больший эл-нт из $arr[i]$, $arr[left]$, $arr[right]$ и его
индекс записывается в переменную largest. Если индекс largest не равен
индексу i, то происходит перестановка эл-ов, таким образов для 
узла $i$ и его дочерних узлов выполняется св-во невозрастания пирамиды, но так как 
теперь $arr[i]$ оказывется на месте $arr[largest]$, то необходимо
рекурсивно вызвать функцию heapify для узла $arr[largest]$.
\\
Заметим, что размер каждого поддерева не превышает величину $2n/3$, причем наихудший
случай осуществляется, когда последний уровень заполнен наполовину. Таким образом,
время выполнения функции описывается рекурентным соотношением$[1]$: 
\begin{center}
    $T(n) = T(2n/3) + O(1)$
\end{center}

А решение данного рекурентного соотношения имеет вид:
\begin{center}
    $T(n) = O(\lg n)$
\end{center}

Далее идет сама пирамидальная сортировка.

\begin{verbatim}
    void heap_sort(long long int* arr, int n) {
        int compare = 0, transp = 0;

        for (int i = n / 2 - 1; i >= 0; i--) {
            heapify(arr, n, i, &compare, &transp);
        }

        for (int i = n - 1; i > 0; i--) {
            swap(&arr[0], &arr[i]);
            transp++;

            heapify(arr, i, 0, &compare, &transp);
        }

        printf(
            "Heap sort, размер массива - %d\n\tКоличество сравнений - %d,
                количество перестановок - %d\n",
            n,
            compare,
            transp
        );
    }
\end{verbatim}

Мы знаем, что каждый вызов heapify стоит нам $O(\lg n)$ и таких вызово будет
$n$ штук, т.е мы можем оценить время работы алгоритма как $O(n\lg n)$. Это верхняя граница, будучи корректной
не является асимптотически точной.
\\

Так же в файле sorts.c (файл где находятся сортировки) находятся следующие вспомогательные функции:

\begin{verbatim}
    // взятие данных из массивов ввода
    long long int* crate_array(int n) {
        long long int* res = (long long int*) malloc(sizeof(long long int) * n);

        return res;
    }

    // перестановка значений двух переменных
    void swap(long long int* a, long long int* b) {
        long long int temp = *a;
        *a = *b;
        *b = temp;
    }

    // вывод массива
    void print_list(long long int* arr, int n) {
        for (int i = 0; i < n; i++) {
            printf("%lld ", arr[i]);
        }

        printf("\n");
    }
\end{verbatim}

Как можно видеть, они просто создают массив, переставляют значения двух переменных и выводят массив. \\

\begin{verbatim}
    long long int* get_data(char* filename, int n) {
        long long int* arr = crate_array(n);
    
        FILE* file = fopen(filename, "r");
        if (file == NULL) {
            printf("Ошибка при чтении файла входных данных для сортировки\n");
    
            return NULL;
        }
    
        for (int i = 0; i < n; i++) {
            fscanf(file, "%lld", &arr[i]);
        }
    
        fclose(file);
    
        return arr;
    }
\end{verbatim}

Данная функция нужна для получения данных из входных файлов.

\begin{verbatim}
    void test(char* filename, int n) {
        long long int* arr = get_data(filename, n);
    
        for (int i = 1; i < n; i++) {
            if (arr[i] < arr[i - 1]) {
                printf("\tОшибка в выходном массиве с количеством эл-ов %d и \n\tпутём %s\n", n, filename);
                
                return;
            }
        }
    
        printf(
            "\tВыходной массив с количством эл-ов %d и
            \n\tпутем %s правильно отсортирован\n", 
            n, 
            filename
        );
    
        free(arr);
    }
    
    // mark - показатель с каким именем сохранять 
    void generate_output(long long int* arr, char* mark, char* sort_type, int n) {
        char filename[MAX_BUF_SIZE];
    
        sprintf(
            filename,
            "%s%s%s_were_%s%d.txt",
            OUTPUT_CATALOG,
            sort_type,
            (strcmp(sort_type, "heap_sort/") == 0 ? "hs" : "sls"),
            mark,
            n
        );
    
        FILE* output_file = fopen(filename, "w");
        if (output_file == NULL) {
            printf("Ошибка при создании файла для вывода данных\n");
    
            return;
        }
    
        for (int i = 0; i < n; i++) {
            fprintf(output_file, "%lld\n", arr[i]);
        }
    
        fclose(output_file);
    
        test(filename, n);
    }
\end{verbatim}

Последняя функция нужна для записи уже отсортированного массива в файлы в определенных каталогах,
в зависимости от типа сортировки и имени для сохранения.\\
Функция \textbf{test} как раз является \textbf{тестирующей функцией}, 
которая проверяет, чтобы все числа в массиве были отсортированный в невозрастающем порядке


\newpage

\section{Результаты экспериментов}

\hspace{5mm} После написания программы и ее запуска на входных тестах, я получил следующие результаты:

\begin{table}[h]
\centering
\begin{tabular}{|c|c|c|c|c|c|c}
    \cline{1-6}
    \multirow{2}{*}{\textbf{n}} & \multirow{2}{*}{\textbf{Параметр}} & \multicolumn{3}{|c|}{\textbf{Номер сгенерированного массива}} & \textbf{Среднее} \\
    \cline{3-6}
    & & \parbox{1.5cm}{\centering 1} & \parbox{1.5cm}{\centering 2} & \parbox{1.5cm}{\centering 3} & \textbf{значение} \\
    \cline{1-6}
    \multirow{2}{*}{10} & Сравнения & 41 & 35 & 40 & 39 & \\
    \cline{2-6}
                        & Перемещения & 30 & 21 & 26 & 26 & \\
    \cline{1-6}
    \multirow{2}{*}{100} & Сравнения & 1081 & 944 & 1027 & 1018 & \\
    \cline{2-6}
                         & Перемещения & 640 & 516 & 581 & 579 & \\
    \cline{1-6}
    \multirow{2}{*}{1000} & Сравнения & 17583 & 15965 & 16798 & 16782 & \\
    \cline{2-6}
                          & Перемещения & 9708 & 8316 & 9066 & 9030 & \\
    \cline{1-6}
    \multirow{2}{*}{10000} & Сравнения & 244460 & 226682 & 235286 & 235476 & \\
    \cline{2-6}
                           & Перемещения & 131956 & 116696 & 124144 & 123266 & \\
    \cline{1-6}
\end{tabular}
\caption{Результаты работы Пирамидальной сортировки}
\end{table}

\begin{table}[h]
\centering
\begin{tabular}{|c|c|c|c|c|c|c}
    \cline{1-6}
    \multirow{2}{*}{\textbf{n}} & \multirow{2}{*}{\textbf{Параметр}} & \multicolumn{3}{|c|}{\textbf{Номер сгенерированного массива}} & \textbf{Среднее} \\
    \cline{3-6}
    & & \parbox{1.5cm}{\centering 1} & \parbox{1.5cm}{\centering 2} & \parbox{1.5cm}{\centering 3} & \textbf{значение} \\
    \cline{1-6}
    \multirow{2}{*}{10} & Сравнения & 54 & 54 & 54 & 54 & \\
    \cline{2-6}
                        & Перемещения & 0 & 5 & 9 & 5 & \\
    \cline{1-6}
    \multirow{2}{*}{100} & Сравнения & 5049 & 5049 & 5049 & 5049 & \\
    \cline{2-6}
                            & Перемещения & 0 & 50 & 98 & 49 & \\
    \cline{1-6}
    \multirow{2}{*}{1000} & Сравнения & 500499 & 500499 & 500499 & 500499 & \\
    \cline{2-6}
                            & Перемещения & 0 & 500 & 990 & 497 & \\
    \cline{1-6}
    \multirow{2}{*}{10000} & Сравнения & 50004999 & 50004999 & 50004999 & 50004999 & \\
    \cline{2-6}
                            & Перемещения & 0 & 5000 & 9992 & 4998 & \\
    \cline{1-6}
\end{tabular}
\caption{Результаты работы Сортировки простым выбором}
\end{table}

\newpage

\section{Отладка программы}

Отладка программы происходила на каждом этапе разрабтки программы:
\begin{itemize}
    \item Написание генератора случайных массивов.
    \item Написание функций для считывания данных из файла.
    \item Написание сортиров и проверка их корректности
\end{itemize}

\subsection{Отладка генератора}

Когда писался генератор массивов и возникали ошибки, то она имела две причины
\begin{itemize}
    \item Ошибка при создании файла, куда класть созданный массив
    \item Ошибка при работе с самим массивом
\end{itemize}

Первая категория ошибок была решена путем создания функции-обертки, которая
при создании файла проверяла, был он создан или нет, и в случае отрицательного ответа она
выводила в консоль сообщение об ошибке, после чего становилось понятно, в чем конкретно 
в программе была ошибка, а не просто констатация завершения программы с каким-то аварийным кодом.
\\
\\
\indentОшибки при работе с массивом решались путем дебагинга программы,
и внимательной работой с индексами.

\subsection{Отладка сортировок}

Отладка сортировок состояла из 
\begin{itemize}
    \item отлаживания функции считывания данных из файла (делалось это через дебагер)
    \item отлаживания факциий сортировок
\end{itemize}

На последнем остановимся поподробней. Так как функции сортировок работают с массивами, то
их отлаживание происходило путем дебагинга программы (когда индекс выходил ха пределы),
а также созданием функции test (была описана выше), которая проверяет результирующий массив на 
правильность (проверяет чтобы числа в нем шли с неубывающем порядке)


\newpage

\section{Асимптотика алгоритмов}

\subsection{Асимптотика пирамидальной сортировки}

Время работы heapify при ее вызове для работы с узлом, который находится на высоте h,
равно $O(h)$, так что общая стоимость процедуры Build Max Heap
\begin{verbatim}
    void heap_sort(long long int* arr, int n) {
        int compare = 0, transp = 0;
    
        for (int i = n / 2 - 1; i >= 0; i--) {
            heapify(arr, n, i, &compare, &transp);
        }
        .....
    }
\end{verbatim}

ограничена сверху значением
\[ 
{\sum_{h=0}^{[\lg n]} \biggl[\frac{n}{2^(h + 1)} \biggr]}O(h) = O\biggl(n\sum_{h=0}^{[\lg n]} \frac{h}{2^h} \biggr)
\]

Последняя сумма вычисляется путем подстановки {$x = \frac{1}{2}$ в формулу, что дает
\[
\sum_{h=0}^{\inf} \frac{h}{2^h} = \frac{ 1/2 }{ (1 - 1/2)^2 } = 2
\]

Таким образом, время работы процедуры Build Max Heap имеет верхнюю границу. 
\[
O\biggl(n\sum_{h=0}^{[\lg n]} \frac{h}{2^h} \biggr) = O\biggl(n\sum_{h = 0}^{\inf} \frac{h}{2^h}  \biggr) = O(n)
\]

Далее становится очевидно, при выполении следующего кода.
\begin{verbatim}
    for (int i = n - 1; i > 0; i--) {
        swap(&arr[0], &arr[i]);
        transp++;

        heapify(arr, i, 0, &compare, &transp);
    }
\end{verbatim}
Мы будем n раз вызывать процедуру heapify, из-за этого
общая сложность сортировки будет равна $O(n\lg n)$. [1]


\newpage

\subsection{Асимптотика сортировки простым выбором}

Так как сама сортировка имеет небольшой размер, то ее можно повторно
вставить сюда, для большей наглядности.
\begin{verbatim}
    void selection_sort(long long int* arr, int n) {
        int j = n - 1;
        int max_i = 0;
    
        int compare = 0, transp = 0;
    
        for (int i = 0; i < n - 1; i++) {
            for (int q = 0; q <= j; q++) {
                compare++;
    
                if (arr[q] > arr[max_i]) {
                    max_i = q;  
                }
            }
    
            swap(&arr[max_i], &arr[j]);
            transp++;
    
            j--;
            max_i = 0;
        }
    
        printf(
            "Selection sort, размер массива - %d\n\t
            Количество сравнений - %d, количество перестаовок - %d\n",
            n,
            compare,
            transp
        );
    }
\end{verbatim}

По циклам видно, что время работы программы зависит от числа эл-ов~$N$,
числа сравнений~$A$ и числа перестановочных максимумов~$B$. Нетрудно видеть, 
что независимо от значений исходных ключей.
\[
A = {n \choose k} = C_{n}^k = \frac{1}{2}N(N - 1)
\]

Следовательно, переменной является только величина~B. Несмотря 
на всю безыскусность простого выбора, не так то легко выполнить точный анализ величины~B.
\[
B = (\min 0,\space ave (N + 1)H_{N} - 2N,\space \max [N^2/4]).
\]

В этом случае становится особенно интересным максимальное значение. Стандартное отклонение
величины B имеет порядок $N^{3/4}$.\\
Такми образом, среднее время работы программы равно
\[
2.5N^2 + 3(N + 1)H_{N} + 3.5N - 11
\]
машинных циклов, т.е данная программа работает лишь немногим медленнее прагораммы, реализующей
метод простых вставок. [2]


\newpage

\section{Анализ допущенных ошибок}

В ходе разработки программы возникали ошибки работы с памятью (забывал чистить память),
которые были устронены. Так же возникали описки в коде, из-за которых в массивах могли генерироваться
одинаковые числа, а не разные. Еще одной ошибкой был неправильный знак сравнения в
пирамидальной сортировке, из-за которого сортировка работала неправильно.
\\
\\
\indentЕще одной ошибкой, наверное самой большой, являлось столь позднее начало 
работы над проектом.

\newpage
\begin{raggedright}
\addcontentsline{toc}{section}{Список цитируемой литературы}
\begin{thebibliography}{99}
\bibitem{cs} Кормен Т., Лейзерсон Ч., Ривест Р., Штайн К. Алгоритмы: построение и анализ.
    Третье издание.~--- М.:<<Вильямс>>, 2013.
\bibitem{cs} Кнут Д. Искусство программирования для ЭВМ.
    Том 3.~--- М.: <<Мир>>, 1978.
\end{thebibliography}
\end{raggedright}


\end{document}
